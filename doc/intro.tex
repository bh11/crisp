%%%%%%%%%%%%%%%%%%%%%%%%%%%%%%%%%%%%%%%%%%%%%%%%%%%%%%%%%%%%%%%%%%%%%%%%%%%%%
%%
%%  construct.tex            CRISP documentation           Burkhard H\"ofling
%%
%%  @(#)$Id$
%%
%%  Copyright (C) 2000, Burkhard H\"ofling, Mathematisches Institut,
%%  Friedrich Schiller-Universit\"at Jena, Germany
%%
%%%%%%%%%%%%%%%%%%%%%%%%%%%%%%%%%%%%%%%%%%%%%%%%%%%%%%%%%%%%%%%%%%%%%%%%%%%%%
\Chapter{Introduction}

\index{CRISP}%

The share package {\CRISP} provides algorithms for computing subgroups of
finite soluble groups related to a group class~$\cal C$. In particular, it
allows to compute $\cal C$-radicals and $\cal C$-injectors for
arbitrary Fitting classes $\cal C$ (and Fitting sets), $\cal C$-residuals
for arbitrary formations $\cal C$, and $\cal C$-projectors for any Schunck
class $\cal C$.

Moreover, the present package contains algorithms for the computation of
normal subgroups belonging to a given group class, including an improved
method to compute the set of all normal subgroups of a finite soluble
group. {\CRISP} also provides basic support for classes (in
the set theoretical sense). The algorithms used are described in
\cite{HoePP1999a}.

$\cal C$-projectors and $\cal C$-injectors of finite soluble groups
arise as generalisations of Sylow and Hall subgroups, and have attracted
considerable interest. They were first studied
by Gasch\accent127utz~\cite{Ga1963}, Schunck~\cite{Sch1967}, and Fischer,
Gasch\accent127utz and Hartley~\cite{FGH1967}. In particular, $\cal
C$-injectors only exist in any finite soluble group if the group class
$\cal C$ is a Fitting class. Similarly, $\cal C$-projectors exist in any
finite group~$G$ if and only if $\cal C$ is a Schunck class. An extensive
account of the subject can be found in~\cite{DH1992}.

In the case when the class $\cal C$ in question is a local formation (which is
a special kind of Schunck class), algorithms for dealing with $\cal
C$-projectors and related subgroups of finite soluble groups are available
also in the {\GAP} share package FORMAT by Eick and Wright; see
also~\cite{EW1999}. In order to use their methods, $\cal C$ has to be
described in terms of algorithms for the computation of residuals with
respect to an integrated local function for $\cal C$.

In {\CRISP}, a group class $\cal C$ is represented by an algorithm
deciding membership in the group class. This information is sufficient to
carry out all operations described in this manual. However, additional
information about the class can be supplied to speed up computations
(sometimes considerably). This information may consist of other classes
(), or of additional algorithms, for instance for the computation of
residuals and local residuals, radicals, or testing membership in related
classes (such as the basis or boundary of a Schunck class).


%%%%%%%%%%%%%%%%%%%%%%%%%%%%%%%%%%%%%%%%%%%%%%%%%%%%%%%%%%%%%%%%%%%%%%%%%%%%%
%%
%E
%%
